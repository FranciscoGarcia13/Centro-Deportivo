% \IUref{IUAdmPS}{Administrar Planta de Selección}
% \IUref{IUModPS}{Modificar Planta de Selección}
% \IUref{IUEliPS}{Eliminar Planta de Selección}

% 


% Copie este bloque por cada caso de uso:
%-------------------------------------- COMIENZA descripción del caso de uso.

%\begin{UseCase}[archivo de imágen]{UCX}{Nombre del Caso de uso}{
	\begin{UseCase}{CU1}{Registrar en el sistema una nueva sucursal.}{
		Almacenar los datos de una sucursal proxima a inaugurarse.
	}
		\UCitem{Versión}{0.1}
		\UCitem{Actor}{Gerente de operación de negocio}
		\UCitem{Propósito}{La sucursal dada de alta en el sistema generara una opcion mas, ya que esta sucursal esta proxima a inaugurarse, a la cual los clientes podran asistir.}
		\UCitem{Entradas}{Los datos de entrada son el nombre de la sucursal, direccion, telefonos, correo electronico, datos referentes al gerente, areas con las que cuenta la sucursal, servicios con los que cuenta la sucursal.}
		\UCitem{Origen}{Los datos de entrada tienen su origen a traves del teclado del computador del actor.}
		\UCitem{Salidas}{Se mostrará un mensaje de registro exitoso en el caso de que no haya errores, si los hay se mostrara un mensaje que informe de dicho error.}
		\UCitem{Destino}{Los mensajes a mostrar se desplegaran en la pantalla del computador del actor. Los datos se enviaran al servidor para procesarlos.}
		\UCitem{Precondiciones}{El actor ingreso al sistema mediante un login. El sistema se encuentra en el formulario, en el cual el actor llenara todos los campos que el formulario muestre.}
		\UCitem{Postcondiciones}{En la base de datos del sistema se tendra un nuevo registro de una sucursal, de las areas con la que esta cuenta, de los servicios que ofrece la sucursal y los datos del gerente de la}
		\UCitem{Errores}{ 1. Los datos proporcionados por el actor no son del tipo especificado en los campos del formulario.

2. La sucursal ya se encuentra registrada.

3. La persona que se especifica con el cargo de gerente de sucursal ya se encuentra como gerente de otra sucursal.

4. Es responsabilidad del actor añadir las areas y servicios con los que realmente cuente la sucursal proxima a inaugurarse. }
		\UCitem{Tipo}{Caso de uso primario}
		\UCitem{Observaciones}{}
		\UCitem{Autor}{Fernández Quiñones Isaac.}
	\end{UseCase}

	\begin{UCtrayectoria}{Principal}
		\UCpaso[\UCactor] Ingresa a la plataforma web e ingresa su usuario y contraseña mediante la \IUref{IU23}{Pantalla de Control de Acceso}\label{CU17Login} para entrar en el sistema.
		\UCpaso Válida que el actor se encuentre dado de alta en el sistema. Se utiliza la regla \BRref{BR117}{Determinar si el usuario tiene acceso al sistema.} \Trayref{A}.
		\UCpaso Despliega la \IUref{IU32}{Pantalla de registro de nueva sucursal} con los campos necesarios para registrar en el sistema la nueva sucursal.
		\UCpaso[\UCactor] Llena todos los campos del formulario y envia el formulario. \Trayref{B}\label{CU17SeleccionarSeminario}.
		\UCpaso verifica los datos introducidos por el \UCactor. \BRref{BR118}{Determinar si los datos de los campos de un formulario son del tipo adecuado} \Trayref{C}.
		\UCpaso Almacena los datos en la base de datos.
		\UCpaso Muestra el \IUref{UI88}{Mensaje de registro exitoso}. 
		\UCpaso Pregunta al estudiante si desea imprimir un comprobante de la inscripción.		
	\end{UCtrayectoria}
		
		\begin{UCtrayectoriaA}{A}{El actor no cuenta con las credenciales validas para poder ingresar al sistema.}
			\UCpaso Muestra el Mensaje {\bf MSG1-}``Usuario [{\em y/o}] contraseñas no validos.''.
			\UCpaso[\UCactor] Oprime el botón \IUbutton{Aceptar}.
			\UCpaso Continua en el paso \ref{CU17Login} del \UCref{CU1}.
		\end{UCtrayectoriaA}
		
		\begin{UCtrayectoriaA}{B}{Alguno de los campos no se especifico o los datos no concuerdan con el tipo esperado.}
			\UCpaso Muestra el Mensaje {\bf MSG1-}``Uno o más [{\em campos}] no tienen el formato adecuado''.
			\UCpaso[\UCactor] Oprime el botón \IUbutton{Aceptar}.
			\UCpaso Pone el foco en el campo donde se encontro el primer error y marca los demas campos con algun error en valor proporcionado por el usuario.
			\UCpaso[\UCactor] corrige el valor erroneo, del campo que tiene el foco y los demas campos con valor erroneo, a un valor correcto.
			      \UCpaso Continua en el paso 5 del del \UCref{CU1}.
		\end{UCtrayectoriaA}
		
%-------------------------------------- TERMINA descripción del caso de uso.
%%%%%%%%%%%%%%%%%%%%%%%%%%%%%%%%%%%%%%

%-------------------------------------- COMIENZA caso de uso para actualizar datos
\begin{UseCase}{CU2}{Actualizar los datos de una sucursal,}{
		Modificar los datos del registro de una sucursal que se encuentran almacenado en la base de datos del sistema.
	}
		\UCitem{Versión}{0.1}
		\UCitem{Actor}{Gerente de operación de negocio.}
		\UCitem{Propósito}{Los datos de una sucursal, tales como los gerente y personal e incluso la direccion de una sucursal, no son permanentes y estos tienden a cambiar. Es por esta razon que se crea el caso de uso \UCref{CU2} para poder actualizar los datos de las sucursales.}
		\UCitem{Entradas}{Los nuevos datos para actualizar un registro son seleccionados por el actor.}
		\UCitem{Origen}{El teclado del equipo de computo del actor.}
		\UCitem{Salidas}{Mensaje de que la modificacion de los datos del regitro de la sucursal seleccionados fue exitosa. Mensaje de error en el caso de que no se llene de manera correcta un campo.}
		\UCitem{Destino}{La pantalla del equipo de computo del actor.}
		\UCitem{Precondiciones}{1. El actor debio haber ingresado al sistema.
		
2. Debe existir el registro a modificar.}
		\UCitem{Postcondiciones}{Se tendra una actualizacion de los datos del registro en la base de datos del sistema.}
		\UCitem{Errores}{}
		\UCitem{Tipo}{Caso de uso primario}
		\UCitem{Observaciones}{}
		\UCitem{Autor}{Fernández Quiñones Isaac.}
	\end{UseCase}

	\begin{UCtrayectoria}{Principal}
		\UCpaso[\UCactor] Entra a la plataforma en linea. Mediante la \IUref{IU23}{Pantalla de Control de Acceso}\label{CU17Login} para entrar en el sistema.
		\UCpaso[\UCactor] proporciona sus credenciales para ingresar al sistema.
		\UCpaso Válida que el actor se encuentre dado de alta en el sistema. Se utiliza la regla \BRref{BR117}{Determinar si el usuario tiene acceso al sistema.} \Trayref{A}.
		\UCpaso Despliega la \IUref{IU32}{Pantalla de sucursales registradas}.
		\UCpaso[\UCactor] selecciona la sucursal a actualizar datos. \label{CU17SeleccionarSeminario}.
		\UCpaso Muestra los datos de la sucursal en un formulario. Este formulario contiene los datos previos a la modificación.
		\UCpaso[\UCactor] modifica los campos necesarios y presiona el boton \IUbutton{Actualizar}. 
		\UCpaso Verifica los datos proporcionados por el \UCactor. \BRref{BR118}{Determinar si los datos de los campos de un formulario son del tipo adecuado} \Trayref{B}.
		\UCpaso Almacena los cambios en la base de datos.
		\UCpaso Muestra el \IUref{UI88}{Datos actualizados satisfactoriamente}. 
		\UCpaso[\UCactor] Presiona el boton \IUbutton{Aceptar}. 
		\UCpaso Pregunta si quiere modificar algun otro registro de una sucursal. \Trayref{C} \Trayref{D}
	\end{UCtrayectoria}
	
		\begin{UCtrayectoriaA}{A}{El actor no cuenta con las credenciales validas para poder ingresar al sistema.}
			\UCpaso Muestra el Mensaje {\bf MSG1-}``Usuario [{\em y/o}] contraseña no validos.''.
			\UCpaso[\UCactor] Oprime el botón \IUbutton{Aceptar}.
			\UCpaso Continua en el paso 1 del \UCref{CU2}
		\end{UCtrayectoriaA}
		
		\begin{UCtrayectoriaA}{B}{Alguno de los campos no se especifico o los datos no concuerdan con el tipo esperado.}
			\UCpaso Muestra el Mensaje {\bf MSG1-}``Un [{\em campo}] no se lleno correctamente.''.
			\UCpaso[\UCactor] Oprime el botón \IUbutton{Aceptar}.
			\UCpaso Pone el foco en el campo donde se encontro el primer error y marca los demas campos con algun error en valor proporcionado por el usuario.
			\UCpaso[\UCactor] Corrige el valor erroneo, del campo que tiene el foco y los demas campos con valor erroneo, a un valor correcto.
			\UCpaso Continua en el paso 5 del caso de uso \UCref{CU2}.
		\end{UCtrayectoriaA}
 
		\begin{UCtrayectoriaA}{C}{}
			\UCpaso[\UCactor] Presiona el boton \IUbutton{Si}.
			\UCpaso Continua en el paso 4 del caso de uso \UCref{CU2}.
		\end{UCtrayectoriaA}
		
		\begin{UCtrayectoriaA}{D}{}
			\UCpaso[\UCactor] Presiona el boton \IUbutton{No}.
			\UCpaso Continua en el paso 12 del caso de uso \UCref{CU2}.
		\end{UCtrayectoriaA}
%------------------------------------- TERMINA caso de uso para actualizar datos de sucursal

%------------------------------------- COMIENZA caso de uso para dar de baja sucursal
\begin{UseCase}{CU3}{Baja de sucursal.}{
		En el caso de clausura o cierre temporal de  una sucursal se dara de baja la sucursal pero no se eliminara el registro de la base de datos.
	}
		\UCitem{Versión}{0.1}
		\UCitem{Actor}{Gerente de operación de negocio}
		\UCitem{Propósito}{Quitar de las tablas que se muestran en la pantalla de los computadores de los clientes, gerentes, intructores, etc. el registro de la sucursal para que ningun usuario trate de acceder a registrar un curso y advertir a los usuarios que la sucursal no esta disponible.}
		\UCitem{Entradas}{La fecha de cierre temporal o clausura de la sucursal, ademas de una descripcion explicando el porque se origina el cierre temporal o clausura.}
		\UCitem{Origen}{El teclado del computador del actor.}
		\UCitem{Salidas}{Se mostrará el mensaje {\bf MSG3-}``La [{\em sucursal}] fue dada de baja.''.}
		\UCitem{Destino}{La pantalla del equipo de cómputo del actor.}
		\UCitem{Precondiciones}{La sucursal no debe estar dada de baja.}
		\UCitem{Postcondiciones}{El sistema tendra una sucursal mas dada de baja. No se mostrara mas este registro a los usuarios, tales como el gerente de sucursal, clientes, instructores, etc. Para el actor gerente de operaciones de negocio si estaran disponibles las sucursales dadas de baja.}
		\UCitem{Errores}{1. La sucursal no se puede dar de baja.

2. No se lleno el campo de descripcion del cierre temporal o clausura.

3. La fecha ingresada es pasada con respecto la fecha en el que se  intenta dar de baja la sucursal en el sistema.}
		\UCitem{Tipo}{Caso de uso primario}
		\UCitem{Observaciones}{La fecha ingresada debe ser la actual o no mayor a un mes, o se puede poner cualquier fecha que no sea pasada a la fecha del registro de la sucursal.}
		\UCitem{Autor}{Fernández Quiñones Isaac.}
	\end{UseCase}

	\begin{UCtrayectoria}{Principal}
		\UCpaso[\UCactor] Ingresa a la plataforma web e ingresa su usuario y contraseña mediante la \IUref{IU23}{Pantalla de Control de Acceso}\label{CU17Login} para entrar en el sistema.
		\UCpaso Válida que el actor se encuentre dado de alta en el sistema. Se utiliza la regla \BRref{BR117}{Determinar si el usuario tiene acceso al sistema.} \Trayref{A}.
		\UCpaso Despliega la \IUref{IU32}{Pantalla de registro de nueva sucursal} con los campos necesarios para registrar en el sistema la nueva sucursal.
		\UCpaso[\UCactor] Llena todos los campos del formulario y envia el formulario. \Trayref{B}\label{CU17SeleccionarSeminario}.
		\UCpaso verifica los datos introducidos por el \UCactor. \BRref{BR118}{Determinar si los datos de los campos de un formulario son del tipo adecuado} \Trayref{C}.
		\UCpaso Almacena los datos en la base de datos.
		\UCpaso Muestra el \IUref{UI88}{Mensaje de registro exitoso}. 
		\UCpaso Pregunta al estudiante si desea imprimir un comprobante de la inscripción.		
	\end{UCtrayectoria}
		
		\begin{UCtrayectoriaA}{A}{El Estudiante no puede inscribir un Seminario}
			\UCpaso Muestra el Mensaje {\bf MSG1-}``El Estudiante [{\em Número de Boleta}] aun no puede inscribirse al seminario.''.
			\UCpaso[\UCactor] Oprime el botón \IUbutton{Aceptar}.
			\UCpaso[] Termina el caso de uso.
		\end{UCtrayectoriaA}



%PLANTILLA
%--------------------------------------
\begin{UseCase}{ID}{Nombre}{
		Descripcion
	}
		\UCitem{Versión}{0.1}
		\UCitem{Actor}{}
		\UCitem{Propósito}{}
		\UCitem{Entradas}{}
		\UCitem{Origen}{}
		\UCitem{Salidas}{}
		\UCitem{Destino}{}
		\UCitem{Precondiciones}{}
		\UCitem{Postcondiciones}{}
		\UCitem{Errores}{ }
		\UCitem{Tipo}{Caso de uso primario}
		\UCitem{Observaciones}{}
		\UCitem{Autor}{Fernández Quiñones Isaac.}
	\end{UseCase}

	\begin{UCtrayectoria}{Principal}
		\UCpaso[\UCactor] Ingresa a la plataforma web e ingresa su usuario y contraseña mediante la \IUref{IU23}{Pantalla de Control de Acceso}\label{CU17Login} para entrar en el sistema.
		\UCpaso Válida que el actor se encuentre dado de alta en el sistema. Se utiliza la regla \BRref{BR117}{Determinar si el usuario tiene acceso al sistema.} \Trayref{A}.
		\UCpaso Despliega la \IUref{IU32}{Pantalla de registro de nueva sucursal} con los campos necesarios para registrar en el sistema la nueva sucursal.
		\UCpaso[\UCactor] Llena todos los campos del formulario y envia el formulario. \Trayref{B}\label{CU17SeleccionarSeminario}.
		\UCpaso verifica los datos introducidos por el \UCactor. \BRref{BR118}{Determinar si los datos de los campos de un formulario son del tipo adecuado} \Trayref{C}.
		\UCpaso Almacena los datos en la base de datos.
		\UCpaso Muestra el \IUref{UI88}{Mensaje de registro exitoso}. 
		\UCpaso Pregunta al estudiante si desea imprimir un comprobante de la inscripción.		
	\end{UCtrayectoria}
		
		\begin{UCtrayectoriaA}{A}{El Estudiante no puede inscribir un Seminario}
			\UCpaso Muestra el Mensaje {\bf MSG1-}``El Estudiante [{\em Número de Boleta}] aun no puede inscribirse al seminario.''.
			\UCpaso[\UCactor] Oprime el botón \IUbutton{Aceptar}.
			\UCpaso[] Termina el caso de uso.
		\end{UCtrayectoriaA}
