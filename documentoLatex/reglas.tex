%---------------------------------------------------------
\section{Glosario de Términos del Negocio}

\begin{description}
	\item[Afiliado: ] Persona menor de edad que necesariamente tiene que estar vínculado con un	
padre o tutor. Pueden ser también personas mayores de edad.	
	\item[Área:] Aula física en la que se imparte un curso.
	\item[Cliente:] Persona que compra una membresía. Ver {\em Usuario final}		
	\item[Horario:] Especificación de días de la semana, hora de inicio y hora de término en los que se imparte un Curso.	
	\item[Membresía:] Permite el acceso a determinadas áreas de cualquier sucursal.
	\item[Usuario final: ] Persona que disfruta de los beneficios de ser miembro de la cadena de gimnasios Deportivo San Pancho. Esta persona puede ser el mismo {\em cliente} o puede ser {\em afiliado} de un cliente.
\end{description}

%---------------------------------------------------------
\section{Hechos del Negocio}

\begin{description}
	\item[Créditos {\em de} un Estudiante:] Es la suma de los Créditos de las Materias aprobadas por el Estudiante.
	\item[Materia {\em aprobada}:] Materia que el Estudiante haya Inscrito y aprobado.
	\item[Materias {\em inscritas} por un Estudiante] Un estudiante se registra a un grupo de una Materia a fin de cursarla y aprobarla.
	\item[{\em Cupo} de un grupo:] Todos los grupos no pueden tener mas de 30 alumnos por la capacidad de las aulas.
\end{description}

%---------------------------------------------------------
\section{Reglas de Negocio}

\begin{BussinesRule}{BR117}{Determinar si el usuario tiene acceso al sistema.} 
	\BRitem[Descripción:] Un Gerente de operación de negocio necesita estar dado de altra en el sistema e introducir su usuario y contraseña de manera correcta.
	\BRitem[Tipo:] Restricción de integridad.
	\BRitem[Nivel:] Obligatorio.
\end{BussinesRule}

\begin{BussinesRule}{BR118}{Determinar si los datos de los campos de un formulario son del tipo adecuado} 
	\BRitem[Descripción:] Los campos pueden ser del tipo cadena de carácteres, valores númericos y valores alfanúmericos
	\BRitem[Tipo:] Restricción de integridad.
	\BRitem[Nivel:] Obligatorio.
\end{BussinesRule}

\begin{BussinesRule}{BR180}{Calcular costos del Estudiante}
	\BRitem[Descripción:] Los servicios se cobran de la siguiente forma:
		\begin{Citemize}
			\item {\em Estudiantes Regulares:} Se les Cobran todos los servicios al 100\% de su costo.
			\item {\em Estudiantes becados:} Se les otorga un 80\% de descuento en el costo de todos los servicios (antes del IVA).
			\item {\em Estudiantes extranjeros:} Se les cobran los servicios al 200\% del costo registrado.
		\end{Citemize}
	\BRitem[Sentencia:] $\forall~e~\in~\mathbb{E}\textrm{studiantes}~\land~\forall~s~\in \mathbb{S}\textrm{eminario}~\Rightarrow$
		\begin{displaymath}
			Costo(e,s) = \left\{ \begin{array}{ll}
			s.costo & , si~e.tipo = \textrm{Estudiante regular}\\
			{s.costo}\over{5} & , si~e.tipo = \textrm{Estudiante becado}\\
			s.costo \cdot 2 & , si~e.tipo = \textrm{Estudiante extranjero}
			\end{array} \right.
		\end{displaymath}

	\BRitem[Tipo:] Cálculo.
	\BRitem[Nivel:] Obligatorio.
\end{BussinesRule}

\begin{BussinesRule}{BR45}{Calcular impuestos por seminario}
	\BRitem[Descripción:] Los impuestos corresponden al 16\% correspondientes al IVA.
	\BRitem[Sentencia:] $Impuesto(e, s) = Costo(e, s)\cdot0.16$.
	\BRitem[Tipo:] Cálculo.
	\BRitem[Nivel:] Obligatorio.
\end{BussinesRule}

\begin{BussinesRule}{BR100}{Recibo del Estudiante por inscripción a Seminario.}
	\BRitem[Descripción:] El  Recibo del Estudiante debe mostrar el total del costo con el siguiente desglose:
		\begin{displaymath}\begin{array}{lr}
			Costo: & \$ XXX.XX\\
			Descuento~aplicado~(YY\%): & \$ XXX.XX\\
			Subtotal: & \$ XXX.XX\\
			IVA~(16\%): & \$ XXX.XX\\\hline
			Total: & \$ XXX.XX
		\end{array}\end{displaymath}
	\BRitem[Sentencia:] $CostoTotal = Costo(e, s) + Impuesto(e, s)$.
	\BRitem[Tipo:] Restricción de operación/Cálculo.
	\BRitem[Nivel:] Obligatorio.
\end{BussinesRule}



